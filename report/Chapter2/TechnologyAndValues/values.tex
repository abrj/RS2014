%!TEX root = ../../report.tex
In recent years their have been a shift in focus within technology design. Designers and developers of Information systems have gone from realizing values in their systems when they have been placed in production at end-users, to now trying to foresee how systems can embed desired values.\newline 
An example of this, is the story of usability. Designers will recall that this value, in the past, was very much a desired featured, but was hardly given a focus in the actual design process. This lack of focus, often lead to information systems which was rejected by end-users, even though the system fulfilled the specified requirements from stakeholders.\newline
The observation here is that it is easy and straightforward to subscribe to the ideal of values(exemplified by usability) in information systems, yet a completely different story to actual use the values actively in the design process.

In order to work with values, a definition of the term \textit{value} are needed. 
\todo{define value} \newline

%\citet{EthicsAndTechnologyDesign} compares ethics, within the field of IT, and Technology and argues that both have experienced a design shift and states: "Ethics has developed from fully theory-oriented through application and context awareness to a focus on the process of designing: similarly IT has gone from being solely technology driven through context awareness to value sensitivity \citep[p. 65]{EthicsAndTechnologyDesign}. This states that designer and developers of Information systems have gone from realizing values in their Systems when they have been placed in production and at the end-users, to now trying to foresee how systems can embed desired values. This topic is also discussed by Nissembaum\todo{not CoRRECT!} \citep{EmbodyingValues}. . \todo{kilde?}.

%A key thing to notice from the above, is that technology and ethics are no longer seen as two independent topics, in regards to information Systems. One can simply not design these, without some sort of ethical impact. This is stressed out by \citet{EthicsAndTechnologyDesign}, that states: ".. it is important to notice that the ethical dimension is not optional; technology has ethical implications regardless of whether the technology has been designed with this in mind or not"\citep[p. 66]{EthicsAndTechnologyDesign}. 

%\subsubsection{Challenges}
%There exists some challenges for embedding values into the design process and moreover the final system. The fact that the idea about values in a design process is a fairly new one, experiences and guidelines for the area is very sparse. A second reason  contributing to the difficulty of embedding values is are more epistemic related issue. The thoughts about values is far away from the a regular design process, and if one wants to incorporate values, a innovative and open-minded perspective is needed from the design team \citep[p. 323]{EmbodyingValues}.
%\todo{Lack of values in traditional developments tools. Cummings p704}


%Some of the challenges designers face today is the fact that previous experience and guidelines, have not yet been developed. The challenge of this process of embedding values can be split in two parts. 
%In todays Information systems a increased focus upon values like privacy, fairness and the like has come to play. As XX\todo{ref til tre skifte} describes the history of the Information Systems can be split into 3 parts. The first part, in the first days, the focus was very much upon the technology it self, how it was used and what it could be used for. In the second part a focus upon things like security and usability became a topic in the development and use of Informations systems, as also suggested by Nissenbaum and Flanagan\todo{p322-kompendium}. In the later (and now present) part the a focus upon softer values has emerged. This part will direct attention to this newly formed focus and draw on inspiration from thinkers like Nissenbaum and XX\todo{another person}. A challenge mentioned, is the fact the spare know-how and knowledge about the actual proceedings for including values in the design process. Another challenge is the fact that known methods and guidelines for how to embody values in the technology is a scarce. Later we present the Value-Sensitive Design approach, which gives a concise framework for how it could be done. 

%\subsubsection{Why Values Matter}
%\citep[p. 65-66]{EthicsAndTechnologyDesign}for this short section a presentation of the current view upon values in system design and why a focus upon this is becoming more and more important, as some has started to say when designing Information systems today a lot of demands and requirements have to be met, from both stakeholders and users. But when designing systems today, compared to 5-10 years ago, a lot of experience can be drawn upon from previous internal or external similar projects in the matter of architecture, functionality and the like. Therefore, when measuring the quality of a system today, and only looking at things like the ones just stated, the systems produced can be seen as being relatively high quality and meeting the technical requirements. But systems produced today has to fulfill more than just the technical demands. More softer issues have to be put in mind of the designers, namely values like privacy, fairness, usability, to mention a few. In order to embed values in a system, designers must begin to have them in mind already at the beginning of the early stages of the design process. Moreover the technical requirements mentioned earlier can already be fulfilled by using experience from other project. but guidelines and experience for embedding values into a design process is more sparse and also very unsaid in development tools today. Popular methods like SCRUM and pair-programming states how a traditionally design process can be conducted, but says nothing about the questions that designers need to ask in order to bring values in to the mind of both developers and designers. 
%\todo{should be deleted?}

\section{Value Sensitive Design}
\label{subsec:vsd_framework}
For reasons as the above, a greater focus upon supporting human values have to come light and have emerged within four areas of design processes: Computer ethics, Social Informatics, Computer Supported Cooperative Work and finally, Participatory Design.\citep[p. 3]{FriedmanVSDandIS}. However, due to certain limitations, which will not be discussed here, within these approaches, another fifth approach have emerged, which: 
\begin{quotation}
\textit{"seeks to design technology that accounts for human values in a principled and comprehensive manner throughout the design process"} \citep[p. 1186]{HumanValuesEthicsAndDesign}
\end{quotation}

This methodology is called Value-Sensitive Design and was developed by Batya Friedmann and Peter Kahn and the key components of this approach will be outline in section \ref{subsec:vsd_framework}.\newline
This design approach is not a game changer in regard to traditionally engineering methods, like Waterfall model or developing method. There a though some similarities between VSD and traditional methods. Many methods begin with some form of design phase, where key concepts are conceptualized and moves on to a analysis and development phase, which are somewhat similar to VSD's technical phase. The project typically concludes with some tests and evaluation phase, similar to the empirical phase of VSD\citep[p. 704]{IntegratingEthicsCummings}.\newline
The VSD approach is meant as a tool and supplement for aiding system designers, toward a more structured way of incorporating ethical concerns and values in to the design process.\newline 
A more detailed description of the different parts of the framework are described in this section.\newline

The Value-Sensitive design framework has identified twelve specific values as which is of great importance in the design of technology: Human welfare, ownership and property, privacy, freedom from bias, universal usability, trust, autonomy, informed consent, accountability, calmness, identity and environmental sustainability. These values have been identified to have ethical import and they should \textit{"have a distinctive claim on resources in the design process"}\citep[p. 1187]{HumanValuesEthicsAndDesign}. Two points should be made about this list of values. Firstly, all the values are not distinctive and could therefore overlap in certain aspects. Secondly, the list is not comprehensive, in the sense, that many other values could possibly be identified to have importance in certain design projects.\newline 
The framework consists of tree parts: Conceptual, Empirical and Technical, which the following subsection describes. 

\subsection{Conceptual}
The first phase of the methodology are an philosophical investigation of the central parts of the system in question. Here questions include: "Who are affected by the system" and "How are certain values supported or dismissed by technological design".\newline 
The goal of this phase is \textit{"... to identify one or two values of central interest that could be viewed as a common thread throughout the project"} \citep[p. 703]{IntegratingEthicsCummings}. Even though it is possible the identified values could be replaced by others during the project period, it is still important to realize them at an early stage, so the awareness of values consists for the entire project. But already here designers must consider different things related to the proposed values. One question to ask, is whether or not a specific value \textit{"... are universal to all humans or pertinent only to local groupings only ..."} \citep[p. 326]{EmbodyingValues}. Designers may quickly agree upon a certain value being universal in their part of the world, say western society, but they still need to question this and maybe even conduct a empirical analysis to justify their initial thoughts. Similar designers should also consider the fact that values can be understood and interpreted differently by different users and a best practice would be to keep to values that is broadly agreed upon. \newline 
Another important aspect of this phase is to identify stakeholders, which can be both direct and indirect. The distinction between the two, are how they interact with the system and how are they affected by the systems output. 

\subsection{Empirical}
The phase that follows is the empirical investigation. This phase is a social investigation of the context of use and the systems impact on the involved stakeholders. Meaning that the focus here is on human interaction. A key focus here is the trade-offs between technical issues and competing values. Here the example of usability can be considered. If one imagined a Internet browser, which gave the user easy access to private browsing history, this would give third party software the same level of easy access, thus potentially compromising the value of privacy.\newline
The phase are described to be used in two ways. The first are in relation to the design making process found in the conceptual phase. Here purely philosophical reasoning could be found insufficient, and one could therefore use a empirical investigating in order to establish which values would be most relevant in the particular setting. This would require designers to explore how the domain area works and how the identified values found in the conceptual phase, would impact stakeholders. \newline
On the other hand, this phase could also be used to establish whether or not a certain value have been embodied successfully and the system is used as intended. To examine the impact a certain value has in the use of a system, a empirical investigation can be conducted after the system has been deployed, by using traditional ethnographic methods of interviewing and observations.
\todo{eksempel på empirisk analyse}

\subsection{Technical}
The last  part of the framework is a technical investigation. The focus is to investigate how certain design decisions either support or dismiss the values identified in the conceptual phase. One question designers can ask is: \textit{"how does this design choice affect the values identified in the conceptual phase?"}. The purpose of this phase is therefore to resolve what kind of values the technology itself is about to embody. Even though the technical and empirical may seem similar they are not, like described by \citet[p. 67]{EthicsAndTechnologyDesign}: \textit{"It should be noticed that technical analyses differ from empirical analyses in that the former focus on technology itself, whereas the latter focus on people and social systems influenced by technology"}\newline
This means that the two phases have entirely different point of interests. The empirical phase looks at the social implications by a technology, whereas the technical phase focuses on the technological mechanisms that can support the values. 


\subsection{VSD: The Practical Way}
\todo{p. 15 \citep{FriedmanVSDandIS}}
