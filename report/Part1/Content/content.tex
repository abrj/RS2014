\section{Content-based Recommender Systems}

\subsection{The functionality of a content-based RS}
The overall reason to implement a Content-based recommender system is basically the same as for other recommender systems; to deliver a list of recommendations that statistically will be seen as valuable to the user. Where it differentiates itself from the others is in the way these recommendations are generated. A content-based recommender system will try to recommend items to the user similar to items that the user has either bought or somehow interacted with in the past. So, in other words, the way the content-based recommender system approaches the recommendation process is by analyzing a set of descriptions of the previously mentioned items which has already been rated or interacted with by the user, and from this information build up a model around the user’s interests. The item preferences from this model will then be compared with those items from the Represented Items repository (see picture). The result returned from the system is a mathematical indication of the relevance these objects or items will represent for the user. As an example, this could help filter search result for webpages. If a resulting webpage has a negative score in comparison to the users interests, it will simply be removed from (or not added to) the list of recommendations.

\subsection{The architecture and modules of the system}
In this section we will describe the different components which substantiates the content-based recommender system by creating the user profile, comparing the user profile with the items, and finally by presenting the recommended items to the user. 
There are three main steps used in the recommendation process, they are described below as follows; content analysis, profile learning, and filtering (see picture).
\begin{itemize}
	\item Content Analyzer: When data has no apparent structure to it, such as text, a processing step which can extract the relevance from the material in a structured way is needed. The main responsibility of the Content Analyzer is to represent data (webpages, articles, documents, etc.) in such a way, that it can be used by the next processing steps. This is done by shifting the content representation from its original form to a form usable by the target module (this could for instance be a webpage represented as keyword vectors) via a specific extraction technique.
	\item Profile Learner: This module collect and utilize the relevant information that has been analyzed and altered into a usable format by the content analyzer. In this case, relevant information is data which represents the users preferences. The profile learner aspires to put this information into the right order, and then use this ordered data to create the user profile for the individual user.
	\item Filtering Component: This module uses the profile representation in question and matches this onto the items in the Represented Items Repository. These items will then be further sorted in relation to the users interests and preferences. The result is generated either as a binary yes or no, or as part of a continuous examination of the item, meaning that perhaps there is not enough information on the user available in order to give a correct judgement of the item, so it is instead put on a “waiting list” until further information on the user can be analyzed.
\end{itemize}
These three components work together to recommend items to the user via a series of steps. These steps are illustrated at the picture below. The first step of the process occurs when the Content Analyzer receives a series of item descriptions from an information source. These descriptions are analyzed in order to extract specific features, such as keywords, from the unstructured text, and end up with a structured representation of the items. These items are stored in the repository Represented Items. In order to create and update the Profile for an active user (Ua) for which the recommendations are required, the system collects information regarding the users behaviour in relation to certain items. This information is then stored in the repository Feedback.
During the Profile Learner step, the above mentioned information in Feedback is processed together with the represented items in order to predict the relevance of these in relation to the user. A user can likewise create a profile where his or her area of interest are directly provided, thus making the Feedback part of optional relevance.
There are basically two types of relevance feedback, positive and negative. The positive feedback indicates features that the user are interested in, whereas the negative feedback is an indicator of features that have no interest to the user. 
These types of feedback can be recorded via two different feedback techniques, namely implicit and explicit feedback. Explicit feedback refer to when the user actively makes a decision, such as a rating or evaluation, about a specific item. Implicit feedback refers to when the user does not make any active involvement, and is captured by monitoring and analyzing the user’s activities and behaviour.
The explicit feedback technique is the easiest way for the Recommender System to make recommendations, since it gives a clear and actual indication of the users preferences. When speaking about the explicit kind of feedback, there are three main approaches: like/dislike, rating, and text comments.
The first option works as a binary scale where the user can either choose to like or dislike an item.
The rating option gives the user a larger scale to rate an item on, this could be from 1-10.
The last option presents the user with a set of text comments from other users in order to aide the user in the decision-making whether to buy or not buy a specific item.\\

All of the above mentioned actions will be stored in the Feedback repository in order to help the Profile Learner create a more comprehensive picture of the users preferences. An advantage of the explicit feedback is that it simplifies the interpretation process of the Recommender System for the specific items in question. A disadvantage can be that the user is required to make an active choice, hence incurring a cognitive load on the user. The implicit feedback is centered around actions that do not require direct user involvement on a cognitive level, but more on their actions in relation to the items in question, such as saving, discarding, bookmarking, etc.\\

In order for the Recommender System to create the User Profile for user Ua, the system must first create a training set from the current repository of reference items. The training set is then used by the Profile Learner to match up against user Ua’s preferences in order to build a solid foundation for the user profile. Future items in the Items repository will be matched against the user item preferences via the Filtering Component and, if there is a match, these items will be shown as recommendations to the user. \\

This training set of recommendations will need to be updated on a regular basis in order to keep the recommendations as close to the users current preferences as possible. This is of particular importance since a users preferences are most likely to be in a dynamic state of change over time.



\subsection{Advantages and disadvantages}
When looking at recommender systems from a content-based systems point of view there are several advantages over a recommender system in a collaborative-based form.\\

The three main areas of advantage are user independence, transparency, and new items.
The user indepence shows itself in that the content-based recommendations are solely based on the active user’s own preferences and does not rely on the interaction of others.
The content-based recommendations are transparent in the way that all its recommendations are clearly based on the list of items placed on the user’s list of preferences, and consists of no estimated guesses. The advantage of a content-based recommender system in regards to new items that enters the Represented Items repository, is that these items will not need to be rated prior to being recommended to the users. Here, the only dependency is whether the item is similar enough to the individual users preferences.\\

As well as the above mentioned advantages, a content-based Recommender System also has some disadvantages. The three main areas here are limited content analysis, over-specialization, and new user’s.
The limited content analysis revolves around the fact, that content-based Recommender Systems most often are dependant on domain knowledge. An example of this could be for a movie recommendation. Here, the Recommender System would need to know the actors, director, genre, and alike. If the Content Analyzer does not find a suitable amount of information, then it will be virtually impossible for the Recommender System to distinguish between items correlating with the user preferences and items that are of no interest at all.\linebreak
The disadvantage with over-specialization is that the user will never be recommended any items in a positive unexpected manner. What is meant by this is that the user’s personal area of preferences might be somewhat wider than the area of the current information held in the User Profile, but since the recommended items for the individual user is based on a match with the items previously rated by the user, there will never be a recommendation deviating from this. This disadvantage is also referred to as The Serendipity Problem.\linebreak
The new user disadvantage can be sort of a predicament for the Recommender System. The issue here is that in order to give the new user a fair recommendation based on user interests and preferences, the user will need to have rated a certain amount of items for the Filtering Component to compare with. When the new user first create his or her account, the recommender system is somewhat forced to either recommend a set of random items or recommend what the average user would be recommended. In most recommender systems the latter is the solution of choice.

%
Where is it used

Use maybe Amazon, and give an example on the occurence of a recommendation, for instance for a book, based on another book that the user has read


How -The mathematical background for the recommendations (different approaches)

Classical and advanced techniques

Different types of algorithms and their function


Trends and directions (what role does user generated content play!?!?)

Next generation of content-based RS

Conclusion








Buzz words:

Serendipitous recommendations
Surprisingly interesting recommendations that the user might not have thought of or discovered otherwise.

Extraction techniques

Information Retrieval systems

- 
%