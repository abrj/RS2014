%!TEX root = ../../report.tex
\section{Discussion}

\subsubsection*{The Process}
\label{subsubsec:process}
The intention of this project was to investigate the areas of recommender systems while also obtaining knowledge within the area of ethics and philosophy.\newline
The first part of the project period was used to examine the area of recommender systems. Here we quickly discovered that the area was more complex than we initially assumed. This meant that we had to limit the scope of the study of recommender systems in order to make room for the area of ethics and philosophy, which was explored during the second part of the project.\newline
Where the area of recommender systems are within the field of mathematics, and thereby our line of study, ethics and philosophy are not. This meant that this area was a particular challenge, because we not only had to examine a great deal of literature, but also understand the language and terms used within this field.\newline
From the beginning, we had an idea that both of these areas would be a challenge to cope with, but we were still surprised with the amount of effort needed in order to comprehend and use both areas in a sensible way in the project. This have resulted in a report which reflects the overall meaning of both areas, but does not present them with the depth that we had initially hoped for.\newline
A better approach could have been to explore the possibilities of the VSD framework within a already known area of software engineering, instead of having to merge two areas, of which we had no prior knowledge, within a project of only 3 months. 
 

\subsubsection*{The 'best' recommender system}
As a starting point for this project, we based the research question upon what a good recommender system is. It is hard to describe what the 'best' recommender system looks like, because the design of a recommender system will always depend on the context of which the system are to be implemented. It is however possible to mention some things, that we believe will help in defining a good recommender system in general. \newline 
For one, a recommender system should not be imposed with any forms of biases. According to section \ref{subsubsec:biasperspective}, this is not always an easy task to accomplish.\newline 
Even though we had the value of freedom from bias in mind throughout the project, we still ended up with some forms of biases. Yet, is it still something that designers should strive to limit as much as possible.\newline
Another thing is the level of relevance. This is very context specific, but a recommender system must always seek to heighten the level of relevance as much as possible. In the case of the ITU system, this have been done by only using an authentic source of feedback in form of history of selected courses.\newline
We could have chosen to include feedback in form of student replies in the course evaluations and possibly giving them the option of rating the courses that they have been recommended, in order to receive a new list of recommendations. These two options of feedback was not included in the system, because we felt that this feedback could be inconsistent, since not all users participate in the course evaluation.


\subsubsection{The ITU recommender system}
The second part of the research question found in section \ref{subsec:researchquestion} asks how a recommender system for courses found at ITU can be made. We have proposed a design for a system in chapter \ref{chap:chapter3}, which reflects the use of the VSD framework together with the knowledge of recommender systems. \newline
Due to reasons explained in section \ref{subsubsec:process}, the design of the ITU system are not as complete and comprehensive as we would have liked it to be.\newline 
On top of these, designing a system without producing the code in parallel with the design, makes it hard to realize how the design choices affects the actual implementation of the system. In our opinion, these two things have had a severe impact on the design of the ITU system. \newline 

Having said that, we do believe that we have reached our initial goal of making a draft for a design, though there is still room for improvement.
\newpage
%\subsubsection*{VSD Framework} Optional\newline
%Using the VSD framework and finding values have been a fun and challenging experience. We started out thinking that it would be a somewhat straightforward linear process, but as it turned out it became more of an iterative approach. We started out like the framework itself suggested, by seeking to identify the relevant values for our project and bring them into the design process. However, this proved to be harder to do than we initially expected. Instead we found ourselves narrowing the list of proposed values from the framework, removing the values we did not feel had an inherent connection to our project. We did this without making any final selections on which of the remaining values to embed. We then continued our approach to design the system, and while doing this we kept the idea of 'value considerations in mind and noted whenever we were faced with a decision which could possibly affect the values in the system. \\
%We believe that it is quite possible to develop a fully functional system without considering using the VSD framework for embedding the values into the system. An alternative to this developing approach could be to simply think of which values to implement in the system, in some cases maybe given in advance by the systems stakeholders. We do feel, however, that the fact that we had a variety of initial values helped us along in the process of both finding the values we felt was important to our system and also to keep one focused on the aspect of embedding values into the system in general.