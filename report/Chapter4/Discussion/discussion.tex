%!TEX root = ../../report.tex
\section{Discussion}
\subsubsection{Next Step}
How would the system be designed if it were to offer seats in courses instead of 'just recommending' courses to users? Or choosing/forcing courses for users, if they do not take courses by them self

We have designed a RS based on similarity, but we started thinking about it as designing it which were based on DIVERSITY? 


What is a good recommender system?
- It might not be a simple question to answer, will always be very context specific



\todo{quote Anders Albrechts Lund hvor han n�vner at designere bruger for lang tid p� at forberede systemet til hvad det m�ske skal kunne, i stedet for at fokusere p� det man ved nu. Det er sv�rt at se hvad der sker i fremtiden}\newline


\todo{describe the affect of deciding not to implementing user interaction into the system: less chance of an emergent bias, less chance of a student who is a teachers assistant being able to get some unfair rights in relation to his or her fellow students.}

\todo{who is a stakeholder and who is not?: We had a discussion about whether a teacher of a course would be both a direct and indirect stakeholder, since they are responsible for adding text to the course descriptions which the system based it structuring of keywords on. }

\todo{do our system take into account that course descriptions can be changed over time? Meaning how do we handle new output from ITU?}