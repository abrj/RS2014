\section{Introduction}
In our everyday life we are constantly exposed to recommendations in some form or the other. Examples of this can be via televised commercials or signs in the supermarket.
It is possible for the television and supermarket advertisers to make more personal recommendations by the use of demographics such as statistics of which age group that watches tv during certain hours and who buys which items in which geographical areas. However, from the group of people that are exposed to these recommendations, it is possible that only a fraction will actually find the commercial relevant to their inherent needs or interests. What if there existed a way to target each individual user with personal recommendations? While we might have to wait a little with that in regard to the television and supermarket (RFIDs are moving us closer to a reality in this area), it does already exist on the Internet and other places: recommender systems.
Where recommendations from the recommender systems described in this paper distinguish themselves from the above mentioned methods, is that the recommender systems are personalizing their recommendations for each of their individual users. This is done by aquiring information about each users preferences and interests, both given directly and indirectly by the user itself. Directly by the user typing in relevant information to the system and indirectly by observing the individual users behaviour via the trail of technological footprints that is left behind when the user interact with the individual site. These footprints includes items bought, items viewed, what persons with similar interests have bought, what they search for, etc. All this information is then crunched in various ways depending on which type of recommender system is used, and recommendations can be presented to the user.

It is argued that we are moving away from an era of search based interaction and entering one of discovery. The meaning  behind this idea is that we are moving away from knowing what we want and hence search for, and instead are shown new products or sites that we did not knew existed, or that we even needed. No matter what, recommender systems will continue to evolve and keep increasing their efficiency, possibly until they can tell more about you than you can yourself.


\todo{Reference this article: \url{http://money.cnn.com/magazines/fortune/fortune_archive/2006/11/27/8394347/index.htm}}

Quote:
"The effect of recommender systems will be one of the most important changes in the next decade," says University of Minnesota computer science professor John Riedl, who built one of the first recommendation engines in the mid-1990s. "The social web is going to be driven by these systems."