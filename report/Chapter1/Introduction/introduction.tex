%!TEX root = ../../report.tex
\section{Introduction}
In our everyday life we are constantly exposed to recommendations in some form or the other. Examples of this are via televised commercials or signs in the supermarket.
It is possible for the television and supermarket advertisers to make more personal recommendations by the use of demographics such as statistics of which age group that watches TV during certain hours and who buys certain items in which geographical areas. However, from the group of people that are exposed to these recommendations, it is possible that only a fraction will actually find the commercial relevant to their inherent needs or interests. What if there existed a way to target each individual user with personal recommendations? While we might have to wait a little with that in regard to the television and supermarket, it does already exist on the Internet and other places and are known as: Recommender Systems.\newline
Where recommendations from the recommender systems described in this paper distinguish themselves from the above mentioned methods, is that the recommender systems are personalizing their recommendations for each of their individual users. This is done by acquiring information about each users preferences and interests, both given directly and indirectly by the user itself. Directly by the user typing in relevant information to the system and indirectly by observing the users behavior via the trail of technological footprints that is left behind when the user interact with a system. These footprints includes items bought, items viewed, what persons with similar interests have bought, what they search for, etc. This information is then used to compute the recommendations that can be presented to the user.

It is argued that we are moving away from an era of search based interaction and entering one of discovery\todo{argued hvor? Find artikel der beskriver dette, en af de sidste vi har l�st}. The meaning behind this idea is that we are moving away from knowing what we want and hence search for, and instead are shown new products or sites that we did not knew existed, or that we even needed. It is likely that recommender systems will continue to evolve and keep increasing their efficiency, possibly until they can tell more about us than we on a conscious level can ourselves.
\todo{lede over til ITU system?}
\newpage

\todo{lede over til ITU recommendations?}
\subsection*{Research Question}
\label{subsec:researchquestion}
In the world today, all users experience recommendations on a daily basis, as suggested above. Recommender systems are embedded into services like Netflix, Amazon, LinkedIn and many other places. \newline
The purpose of this project is to get a deeper understanding of how recommender systems work and to suggest a recommender systems design for the IT-University, as well as explore the possibilities of designing with human values in mind. \newline \todo{what about philosophical?}
The following questions will be discussed and possibly answered: How do Recommender Systems work? Do they all work the same way or are different types of recommender systems in play? What role does a philosophical perspective play in the making of a recommender system? However, the main purpose of this report are to answer the following: \newline

\textbf{What defines a good recommender system and how can a recommender system for courses at ITU be made?}\newline

The answer to this question, will be found by following the structure presented below.
\todo{more here?}
\subsection*{Overview}
This section are meant to provide the reader with a overview of the structure of the report.\newline
The report is overall split into four chapters, which also reflects how the work process have been throughout the project.\newline 
The first chapter is about the different types of recommender system and how they work. This part describes two types of recommender systems and ways to make hybrid versions between the two.\newline
The second chapter represents the philosophical part of the project. This section describes a framework, that we have tried to use in the design process for the recommender system for ITU. \newline
The third chapter reflects our conception of the recommender system for courses at ITU. This part of the report have been produced by using the knowledge obtained from chapter 1 and 2.\newline
Chapter four concludes the project and report and does therefore contain a discussion and reflection of the proposed system and the project in general.
\newpage