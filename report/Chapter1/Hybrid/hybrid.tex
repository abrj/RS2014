%!TEX root = ../../report.tex
\newpage
\section{Hybrid approaches}
So far, descriptions of collaborative and content-based recommender systems have been given in sections \ref{sec:collaborative} and \ref{sec:content}. These two have, of course, both advantages and disadvantages as also have described earlier. To overcome some of these disadvantages, one could combine the approaches into a 'hybrid' recommender system. This could for instance be in a case, where a system have a rich community of users and the items in that system could be compared by keywords, then recommendations could be made by combining the features of content-based recommendations with the collaborative based approach of looking at peer-users, which, in theory, could result in higher quality recommendations.
In the following a list of applied hybridized approaches will be presented, to outline the different methods for doing this:
According to \citet[p. 20-22]{TowardsTheNextGenerationOfRs} the methods can be classified as the following four: "(1) implementing collaborative and content-based methods separately and combining their predictions, (2) incorporating some content-based characteristics into a collaborative approach, (3) incorporating some collaborative characteristics into a content-based approach, and (4) constructing a general unifying model that incorporates both content-based and collaborative characteristics". It should be noted here, that this list is probably not comprehensive and many, many other hybridized approaches exists. \newline
The first method is to implement the collaborative and content-based independently of each other and then obtain a recommendation for the same item from both, and then afterwards either combining the rating from each system into a single rating or selecting the 'best' rating relative to the recommended item based on a predetermined quality setting. \newline
The second method is to add some content-based characteristics to a collaborative recommender system. Here the system uses the 'Profile learner' to compare the similarity between users and thereby partly overcomes some of the sparsity problems, since user-pairs most often does not have many ratings in common. Another feature of this approach is that users can be recommended an item based on, not only if it is highly rated by peer-users, but also if this item matches the users profile. \newline
The third method used some kind of dimensionality reduction technique on the profiles given from the 'Profile Learner' in the content-based recommender approach. In the simple terms, this mean reducing the number of dimensions found in the vector space, in order to improve performance compared to the pure content-based approach. \newline
The fourth hybrid approach is the combination of the collaborative and content-based in to one single recommendation model. The approach have been adopted by many researchers in recent years and because of this, many different methods of doing this hybrydizing have been proposed \citep[p. 22]{TowardsTheNextGenerationOfRs}. These different methods somewhat lyes outside of the scope of this report and will therefore not be further described. \newline

