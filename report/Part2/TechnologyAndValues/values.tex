%!TEX root = ../../report.tex

\section{Technology and values}
Is this part of the report a focus upon values in technology will be placed. In \todo{Ethics and technology design} the author compares ethics and Technology and argues that they both have experienced a design shift and states that: "Ethics has developed from fully theory-oriented through application and context awareness to a focus on the process of designing: similarly IT has gone from being solely technology driven through context awareness to value sensitivity". This states that designer and developers of Information systems have gone from realizing values in their Systems when they have been placed in production and at the end-users, to now trying to foresee how systems can embed desired values. This topic is also discussed by Nissembaum\todo{Kilde: Embodying values in technology}. The observation here is that it is easy and straigtforward to subscribe to the ideal of values in Information Systems, yet a completely different story to actual use the values actively in the design process. An example is the story of usability. Designers will recall that this value, in the past, was very much a desired featured, but was hardly given a focus in the actual design process. This lack of focus, often lead to Information Systems which was rejected by end-users, even though the system fulfilled the specified requirements from stakeholders\todo{kilde?}. In present times companies and branches of sofware development have emerged, with a sole focus upon usability\todo{kilde?}. Likewise, the value of privacy for online systems have been a topic for debate in the last couple of years\todo{kilde?}

The challenge of this process of embedding values can be split in two parts. 

In todays Information systems a increased focus upon values like privacy, fairness and the like has come to play. As XX\todo{ref til tre skifte} describes the history of the Information Systems can be split into 3 parts. The first part, in the first days, the focus was very much upon the technology it self, how it was used and what it coould be used for. In the second part a focus upon things like security and usability became a topic in the development and use of Informations systems, as also suggested by Nissenbaum and Flanagan\todo{p322kompendium}. In the later (and now present) part the a focus upon softer values has emerged. This part will direct attention to this newly formed focus and draw on inspiration from thinkers like Nissenbaum and XX\todo{another person}. A challenge mentioned, is the fact the spare know-how and knowledge about the actual procedings for including values in the design process. Another challenge is the fact that known methods and guidelines for how to embody values in the technology is a scarce. Later we present the Value-Sensitive Design approach, which gives a concise framework for how it could be done. 

\subsection{Why Values Matter}
for this short section a presetation of the current view upon values in system design and why a focus upon this is becoming more and more important, as some has started to say
when designing Information systems today a lot of demands and requirements have to be met, from both stakeholders and users. But when designing systems today, compared to 5-10 years ago, a lot of experience can be drawn upon from previous internal or external similar projects in the matter of achitecture, functionality and the like. Therfore, when measuring the quality of a system today, and only looking at things like the ones just stated, the systems produced can be seen as being relatively high quality and meeting the technical requirements. But systems produced today has to fulfill more than just the technical demands. More softer issues have to be put in mind of the designers, namely values like privacy, fairness, usability, to mention a few. In order to embed values in a system, designers must begin to have them in mind already at the beginning of the early stages of the design process. Moreover the technical requirements mentioned earlier can already be fulfilled by using experience from other project. but guidelines and experience for embedding values into a design process is more sparse and also very unsaid in development tools today. Popular methods like SCRUM and pair-programming states how a traditionally design process can be conducted, but says nothing about the questions that designers need to ask in order to bring values in to the mind of both developers and designers. 

\subsection{Value Sensitive Design}
A method for reaching the proposed thoughts above is known as Value Sensitive Design (VSD) and was suggest by Nissenbaum\todo{kilde?} in \todo{kilde til artikel}. The framework has identified twelve specific values as which is of great importance for designing Informations Systems. The framework consists of tree parts: Conceptual, Empirical and Technical.
The goal of the first phase "would be to identify one or two values of central interest that could be viewed as a common thread throughout the project". Even though it is possibly the identified values could be replaced by others during the project period, it is still important to realize them at an early stage, so the awareness of values consists for the entire project. 
The phase that follows is the empirical investigation. A key focus here is the trade-offs between technical issues and competing values. To examine the impact a certain value has in the use of a system, a investigation can be conducted after the system has been deployed. 
The last  part of the framework is a technical investigation. Here questions like: "how does this design choice affect the values indetified in the concetual phase?". 

