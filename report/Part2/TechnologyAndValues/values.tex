%!TEX root = ../../report.tex

\section{Technology and values}
Is this part of the report a focus upon values in technology will be placed. In todays Information systems a increased focus upon values like privacy, fairness and the like has come to play. As XX\todo{ref til tre skifte} describes the history of the Information Systems can be split into 3 parts. The first part, in the first days, the focus was very much upon the technology it self, how it was used and what it coould be used for. In the second part a focus upon things like security and usability became a topic in the development and use of Informations systems. In the later (and now present) part the a focus upon softer values has emerged. This part will direct attention to this newly formed focus and draw on inspiration from thinkers like Nissenbaum and XX\todo{another person} 

\subsection{Why Values Matter}
for this short section a presetation of the current view upon values in system design and why a focus upon this is becoming more and more important, as some has started to say
when designing Information systems today a lot of demands and requirements have to be met, from both stakeholders and users. But when designing systems today, compared to 5-10 years ago, a lot of experience can be drawn upon from previous internal or external similar projects in the matter of achitecture, functionality and the like. Therfore, when measuring the quality of a system today, and only looking at things like the ones just stated, the systems produced can be seen as being relatively high quality and meeting the technical requirements. But systems produced today has to fulfill more than just the technical demands. More softer issues have to be put in mind of the designers, namely values like privacy, fairness, usability, to mention a few. In order to embed values in a system, designers must begin to have them in mind already at the beginning of the early stages of the design process. Moreover the technical requirements mentioned earlier can already be fulfilled by using experience from other project. but guidelines and experience for embedding values into a design process is more sparse and also very unsaid in development tools today. Popular methods like SCRUM and pair-programming states how a traditionally design process can be conducted, but says nothing about the questions that designers need to ask in order to bring values in to the mind of both developers and designers. 

\subsection{Value Sensitive Design}
A method for reaching the proposed thoughts above is known as Value Sensitive Design (VSD) and was suggest by Nissenbaum\todo{kilde?} in \todo{kilde til artikel}. The framework has identified twelve specific values as which is of great importance for designing Informations Systems. The framework consists of tree parts: Conceptual, Empirical and Technical. The goal of the first phase "would be to identify one or two values of central interest that could be viewed as a common thread throughout the project". Even though it is possibly the identified values could be replaced by others during the project period, it is still important to realize them at an early stage, so the awareness of values consists for the entire project. 
The phase that follows is the empirical investigation. A key focus here is the trade-offs between technical issues and competing values. To examine the impact a certain value has in the use of a system, a investigation can be conducted after the system has been deployed. 
The last  
