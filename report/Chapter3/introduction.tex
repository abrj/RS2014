%!TEX root = ../report.tex
So far the report have mainly been focused upon the different types and methodology of recommender systems along with the theory for values in design. At this point, the reader should have been provided with an overview of the two topics. In this third chapter the objective is to use and apply the obtained knowledge from chapter \ref{chap:chapter1} and \ref{chap:chapter2} to design a recommender system, with some desired values in mind.\newline 
A recommender system can be seen in many forms and be found many places, as described in chapter \ref{chap:chapter1}. For this report, a recommender system for courses at the IT University, will be conceived. The reason for this, is that all full-time students are faced with one or more options for non-mandatory courses throughout their education. This can, for some students, be hard to do, because of the many courses available as well as the lack of transparency of the content for each of these courses.\newline
It is important to note that the proposed system will not be build, but only be designed at a fitting layer of abstraction. The most natural step would of course be to create the design along side the development of the system. There are two reasons for not doing this. The first is a pure time related issue. This project was done over a period of approximately 3 months, where the primary focus has been to read the needed literature to present a fitting perspective upon the topics of recommender system techniques and values in technology. The second reason for not including an implementation stage are grounded in the process. A popular thought in these times are the one about agile and iterative development methods. Here developers wants to move away from the more traditionally waterfall method and into a more agile method, which in some cases means coding while designing. This is a natural process, which converts the design decisions into code almost at the same time. This way there is a short way to prototypes. To do this, the design process has to reflect the levels of details needed to implement the decisions. The code will then end up reflecting the designed intentions, but at some point it is very likely that the code will begin to impact the design process, which is exactly why this methodology is so strong. Design and code is correlated and influences each other. Because of the time restraint, we believed the most appropriate approach, would be to focus only on the design process and exclude the implementation This way we would be able to work in a linear process and focusing on the best way to embed the desired values, without having to worry about the details of the implementation.\\
This means that the following section will go as much into detail as possible for the design process, without worrying about the low level details of implementation and production. With this in place, we will continue with the design of the system. 

\newpage
% 0.5-1 page
\section{ITU Domain}
To design the system it is needed to explore the domain in which the system will be designed. To get an overview of this, this section will explain the different terms which are used and an explanation of what kind of educations the IT University offers their students.\newline
The educations is divided into 3 sections: Bachelor, Master and Diploma. Both the Bachelor and Master are full time studies, whereas the Diploma is targeted for people who are already working and wish to complement their education. There are three types of bachelors tracks: Global Business Informatics (BGBI), Software Development (BSWU) and Digital Media and Design (BDMD). The Masters are divided into: Digital Innovation \& Management, Digital Design \& Communication, Software Development and Technology and lastly Games. Under the Master programs the students can choose different tracks (also called specializations), but these will not be taken into consideration in the system, and are therefore not explained here. The Diploma is designed to be a part time education, with single elective courses. Here it is also possible to choose tracks, but like for the Masters, these will not be further explained. 
A bachelor education consists of 180 ECTS points, a Master of 120 ECTS points and a Diploma of 60 ECTS points. Since the Bachelors consists of many mandatory courses, not a lot of ECTS points are available for elective courses. The direct opposite is the case for the Masters. Here most of the 120 ECTS points are elective, which is the same for the Diploma. \newline
The different people associated with ITU can be divided into three groups: Students, Researchers and Staff. The most interesting group is of course the students, because this is the group for which the recommendations will be targeted. This group can be further divided into full- and part-time students. Most people on the Bachelors and Masters programs are full-time students, whereas a small part of Masters as well as all people on Diploma are part-time for various reasons. 
\newpage