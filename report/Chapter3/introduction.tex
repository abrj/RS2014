%!TEX root = ../report.tex
So far the report have mainly been focused upon the types and methology of different types of recommender systems and the theory for values in design. This should at this point of reading, have provided the reader with a overview of the two topics and a little feel of the relation between the two. For this third the objective is to use and apply the obtained knowledge from chapter \ref{chap:chapter1} and \ref{chap:chapter2} to design a recommender system, with some desired values in mind. It is important to note, that the proposed system will not be build, but only be designed at a fitting layer of abstraction. The most natural step would of course be to design, and along the side, implement the chosen design. The reasoning for not including this, is a two split decision. The first is a pure time related issue. This project was done over a period of approximately 3 months, where the primary focus have been upon reading the needed literature in order to present a fitting perspective upon the topics of recommender system techniques and values in technology. The second reason for not including a implementation stage are grounded in the process. A popular thought in these times are the one about agile and iterative development methods. Here developers wants to move away from the more traditionally waterfall method and into a more agile method, which in some cases means code-while-designing. This is a natural process, which converts the design decisions into code almost at the same time. This way there is a short way to prototypes. In order to do this, the design process has to reflect the levels of details needed to implement the decisions. The code will then end up reflecting the designed intentions, but at some point it is very likely that the code will begin to impact the design process, which is exactly why this methodology is so strong. Design and code is correlated and influences each other. Because of the time restraint, we believed the must appropriate approach, would be to focus only on the design process and excluding the implementation This way we will be able to work in a linear process and focusing on the best way to embed the desired values, without having to worry about the details of the implementation.\\
This means that the following section will go as much in to detail as possible for the design process, without worrying about the low level details of implementation and production. With this in place, it is possible to continue. 


% 0.5-1 page
\subsection{ITU Domain}
In order to design the system it is needed to explore the domain in which the system needs to be designed. To get an overview of this, this section will be used to explain the different terms which is used and explanation of what kind of educations the IT University offers students. Some words used here will be domain specific, but will be explained in the Glossary \ref{glossary}.
To start at the highest level, the educations is divided into 3 sections: Bachelor, Master and Diploma. Both the Bachelor and Master are full time studies, whereas the Diploma is targeted for people already working and wants to complement their education. There is three types of bachelors tracks: Global Business Informatics (BGBI), Software Development (BSWU) and Digital Media and Design (BDMD). The Masters are divided into: Digital Innovation \& Management, Digital Design \& Communication, Software Development and Technology and lastly Games. Under the Masters different tracks (also called specializations) can be chosen, but these will most likely not be taken into considerations in the system, and therefore not explained here. The Diploma is designed to be a part time education, with single elective courses. Also here, it is possible to choose tracks, but like the Masters, will these not be explained. 
A bachelor education consists of 180 ECTS points, a Master on 120 ECTS points and Diploma 60 ECTS points. Since the Bachelors consists of many mandatory courses, not a lot of ECTS points a available for elective courses. The direct opposite is the case for the Masters. Here most of the 120 ECTS points are elective, which is the same for the Diploma. \newline
The different people associated with ITU can be divided into three groups: Students, Researchers and Staff. The most interesting group is of course the students, since this in this group, the recommendations will be targeted. This group can be further divided into full- and part-time students. Most people on the Bachelors and Masters are full-time whereas a small part of Masters as well as all people on Diploma are part-time, for different reasons. 
