%!TEX root = ../../report.tex

 
\section{Technical}

% 0.5 page
\subsection{Introduction}
As a part of the VSD framework, a technical part of the recommender system is required. In this chapter of the report, a focus will therefore be on the actual domain where the system is to be implemented, the technological choices taken for the systems. This will be done with the chosen values from the conceptual section in mind and grounded in these values. 
The section are divided in to two parts. The first will be an analysis, whereas the second will focus upon the implementation and challenges for doing so. 
A recommender system can be seen in many forms and be found many places, as described in section \todo{ref til section}. For this report, a recommender system for courses at the IT University, will be conceived. The reason for this, is that all full-time students are faced with one or more options for non-mandatory courses throughout their education. This can, for some students, be hard to do, because of the many courses available aswell as the lack of transparency of the content for each of these courses.


% 0.5-1 page
\subsection{Domain Description}
In order to design the system it is needed to explore the domain in which the system needs to be designed. To get an overview of this, this section will be used to explain the different terms which is used and explanation of what kind of educations the university offers students. Some works used here will be domain specific, but will be explained in the Glossary found in Appendix \todo{REF TIL GLOSSARY}
To start at the highest level, the educations is divided into 3 sections: Bachelor, Master and Diploma. Both the Bachelor and Master are fulltime studies, whereas the Diploma is targeted for people already working and wants to complement or upgrade their education. There is three types of bachelors tracks: Global Business Informatics (BGBI), Software Development (BSWU) and Digital Media and Design (BDMD). The Masters are divided into: Digital Innovation \& Management, Digital Design \& Communication, Software Development and Technology and lastly Games. Under the Masters different tracks (also called spezialations) can be chosen, but these will most likely not be taken into considerations in the system, and therefore not explained here. The Dimploma is designed to be a part time education, with single elective courses. Also here, it is possible to choose tracks, but like the Masters, will these not be explained. 
A bachelor education consists of 180 ECTS points, a Master on 120 ECTS points and Diploma 60 ECTS points. Since the Bachelors consists of many mandatory courses, not a lot of ECTS points a available for elective courses. The direct opposite is the case for the Masters. Here most of the 120 ECTS points are elective, which is the same for the Diploma. 
The different people associated with ITU can be divided into three groups: Students, Researchers and Staff. The most interesting group is of course the students, since this in this group, the recommendations will be targeted. This group can be further divided into full- and part-time students. Most people on the Bachelors and Masters are full-time whereas a small part of Masters aswell as all people on Diploma er part-time, for different reasons. 
The three types of educations all have elective courses, but the the Masters have more 
What domain are we in? Description of special words. Explanation of the different educations/track/etc

% 1-2 page
\subsection{Design Choices}
In this section explanations of the design choices made throughout the design process will be presented aswell as the 'consequences' of choosing the different solutions. 
\subsubsection*{Collaborative vs Content}
One of the first and most important choices made for this system, is of course what kind of recommendation type of system, that should be developed. In section \todo{ref til chapter 1} we outlined three different kinds of types of recommeder systems: Collaborative-filtering (CF), Content-based (CT) and Hybrid approaches. As also explained in \todo{ref til chapter 1/section} the two first types somewhat represents extremes and it is most common to use a hybrid of the two. To keep the system as simple as possible for a start, the approach chosen is the content-based one. The reasons for this, will be outlined below. \\
The CF approach mostly builds upon ratings from users, which is used to either compare users with users or items with items, and then produce a list of recommendations. This makes the ratings a central and crucial part of the system. In order to implement this approach for a ITU recommender systems, these ratings could come from the course evaluation, which basically asks the students to 'rate' and comment on both the course and teacher. This could definitely be a viable solution, except that it relies a great deal upon the data from the course evaluation. Unfortunately the evaluation is only done by around 35-40\% students, which means the system will be affected of the things described as the 'Data sparsity' problem in section \todo{ref til data sparsity}. 
Target group?
Tidsplan?
Persistence or memory?
% 0.5-1 page
\subsection{Data Modelling}
ER-Diagram

% 0.5-1 page
\subsection{Architecture Analysis}
Flow of the program
Picture of package diagram/components

% ? page
\subsection{Glossary}
\begin{itemize}
	\item[ECTS points] 
	\item[SWU]
	\item[GBI]
	\item \todo{more?}
	\item[CF] Collaborative-Filtering approach
	\item[CT] Content-based approach
	\item[Course evaluation]
\end{itemize}
% 1-2 page
\subsection{ITU Recommender System}
Profile Learner
Content Analyzer
Filtering Component


%#############% 
\subsubsection{Data on students}
First step is to divide up the students into the groups where they belong, using the first five student data points.

- Education
- Track
- Semester (Bachelor students can only choose an elective course on 4th and 5th semester) (Masters students can choose an elective course on each semester (4th is only their thesis))
- ECTS points earned
- Nationality

Only when the students have been divided up into groups is where the recommender system will go to work. By analyzing the next student data points the RS will recommend elective courses to the students in question.

(which courses has the student already had, (DO NOT TAKE COURSE AND EXAM DATES INTO CONSIDERATION (AMAZON DOES NOT DECIDE WHETHER SOMETHING IS TOO EXPENSIVE FOR ME OR WHETHER I NEED THAT, THAT IS UP TO THE INDIVIDUAL TO DECIDE, THE SAME APPLIES HERE)))

- Grades (only in relation to elective courses?)
- Previous grades (all courses are available to all students that meets the formal requirements) (just because you get a low grade on a previous course does not necessarily makes that person less interested in a course where that previous course is a pre-requisite)
- Previous courses (mandatory) and projects
- Previous elective courses (could be a sign of interest)
- Age (when you have been accepted to ITU you are in and can freely choose which courses you wish to attend no matter of age - we do not want to implement any bias into the system)
- Current courses (to make sure that the recommended elective course is not time wise placed on top of on of the mandatory courses)

The last two student data point serves as contact information (name is also used to refer to the above mentioned data points).
 
- Name
- Mail
- 

\subsubsection{Course information}

- Name
- ECTS (should be included, so a student with only 7,5 etcs available, will not be recommended courses with 15 etcs)
- Which semester
- Course code
- Course language
- Max seats
- Current number of taken seats
- Formal requirements (to vaguely defined in order to extract proper information)
- Course learning goals (not important)
- Academic content (this is here the keywords are extracted for the recommender system)
- Mandatory activities (not important)
- Exam forms (date and description) (the student must check if exam dates collide)
- Litterature (not important)
- Curricullum (not important)
- Course responsible/manager (not important)
- Relevant for (needs to properly defined, but is usefull)
- Time and place (it is the students responsible to check if courses collide)


\subsubsection{Persistence vs memory}
even though the program only will run 2-3 times per year, perstiente the data to a database is needed. One could simpy keep the data in memory, but then it would be needed to 'reprofile' all students and analyse the content for each course, for each run of the program. When perstiente the data, the program 'only' needs to compare the new data with the previous created profiles and the content for the courses.

\subsubsection{List of Requirements}
\begin{itemize}
	\item The list of possible courses for each student, must NOT contain mandatory courses or courses the student already have had.
	\item Der skelnes mellem tilmeldte og tidligere tilmeldte kurser, i forhold til bestået kurser, for at undgå at man bliver anbefalet kurser, som man er igang med på tidspunktet hvor anbefalingen falder. 
\end{itemize}
  

\subsubsection{Recommender system output}

- Max seats in recommended course
- Current number os seats taken in recommended course
- NOTE: please check that the recommended courses are not colliding with your mandatory or other elective courses
- 




FIND INFORMATION THAT DESCRIBES HOW A CONTENT ANALYZER, PROFILE LEARNER AND FILTERING COMPONENT ARE 

MAKE NEW DRAWING SHOWING THE INFORMATION FLOW WITHIN THE SYSTEM BOTH FOR CONTENT-BASED AND COLLABORATIVE



IF COLLABORATIVE BASED (ALWAYS RATING BASED): USER-USER OR ITEM-ITEM

It will be a choice between what the course contain (can't really be shown from the users ratings of the course, since two courses can be completely different in regard to content, but have the same overall ratings) and how popular the course is (can be shown by the ratings, but will be a problem with new courses). Only about 40 percent of students are rating their courses via the course evaluation.
\subsubsection{Discussion}
The system is requiring feedback from users. The only feedback available is on previous selected courses, which is not a lot. But this is in general a problem for many recommender systems, the lack of feedback from users, being both ratings and content-feedback. 