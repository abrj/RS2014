%!TEX root = ../../report.tex
\section{Empirical}
According to the VSD framework, a part of embedding values into a system is to investigate what values the system should pay attention to, examine how these values are treated by future users of the system, and to see if the intended values do in fact 'impact' the users and intended. This means that the phase could be used both before the technical phase, but also after to see how the intended values are 'received' by the users. \newline
Due to the time constraint described in the introduction to this chapter, this phase have not been completed. Instead, for this project, the phase have been used to try to imagine what questions and areas that needed answer in order to design the system. This have lead us to two topics that needs investigation: 1) the ITU domain and 2) the intentions and thoughts behind the courses, educations and tracks. The first is a purely a questions about understanding the domain that the system needs to be produced in and for. Questions like "How many bachelors lines are there at ITU and what do each line hold?". The second is more related to the purpose of courses and different lines. A question here could be "Why does the bachelor lines have both elective and mandatory courses?"

This section could be about some data gathering from leading people from ITU, who has something to say about the student profile, when graduating from ITU. It could also be something about what leading people from ITU says in general about the students choice of optional courses and the purpose of having optional courses at ITU. 




