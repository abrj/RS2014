%!TEX root = ../../report.tex
\section{Conceptual}

\todo{move this part}describe stakeholders, benefits/harms, and value identification in the VSD section. Move the text below to the part in chapter 2 where if fits in\newline

Stakeholders consists of individuals or groups that are directly or indirectly interacting with or affected by the system. Direct stakeholders are those individuals or groups that are interacting with the system in a direct manner or are subsequently affected by or interacting with the output of the system. Indirect stakeholders are the people who are affected by the system though they never directly interact with it. It is possible for an individual to be both directly and indirectly affected by the system, and hence being both a direct and indirect stakeholder.

Benefits and harms defines the good and bad sides that the system have for the stakeholders of the system. In \citep[p.12]{FriedmanVSDandIS} it is described that it might be possible in some way to make every human an indirect stakeholder of the system, but here stresses the importance of prioritizing the individuals or groups that are most affected on an indirect level. 


Identifying the values that are affected by the benefits and harms are done to help determine whether a specific value has been successfully implemented into the system. 

------------



In accordance with the VSD-framework, this section we will describe how we practically identified the stakeholders for the system along with the benefits and harms for these stakeholders. We will then continue to define which impact these benefits and harms will have on our values.

Benefits and harms defines the good and bad sides that the system have for the stakeholders of the system.


In accordance with the VSD-framework, this section we will describe how we practically identified the stakeholders for the system along with the benefits and harms for these stakeholders. We will then continue to define which impact these benefits and harms will have on our values.



Stakeholders for the ITU recommender system:
In order to find our stakeholders, we sat down and tried to identify all people that were interacting with or affected by the system in any way. Below are the results we came up with.

Direct stakeholders:  These stakeholders include:
- Students. The students are interacting with the output of the system, and can be further divided into subgroups: Full time students (bachelors, masters) and part time students (bachelors, masters, single course students).
- Teachers. Our initial thought was that the teachers are responsible for the course descriptions which defines the keywords for the individual courses, and hence have an impact on the system. We have subsequently discussed whether a teachers actually were a direct stakeholder, since our system only collects the data information from ITUs servers, and the teachers therefore do not have any direct interaction with the system. The reason we finally decided that they are part of the direct stakeholders is, that by changing the course description they will have the opportunity to alter the output of the system, hence we believe they should be seen as a direct stakeholder)
- Persons responsible for producing the non-keyword list. They have are impacting the system in relation to the non-keywords they type in, since these can potentially alter the outcome of a keyword analysis, hence changing the relevance of a specific course.

Indirect stakeholders: 
- The teachers. The teachers are indirectly impacted in regards to whether students choose to sign up for their course(s) or not, since the course can potentially be cancelled if there are not a minimum number of sign-ups.
- Teachers assistants. Impacted indirectly in same way as teachers.
- IT University of Copenhagen. All the students can freely choose where they want to take their elective courses. ITU is indirectly impacted in the way that  through the system the students are encouraged to choose elective courses from within ITU. 


The above show us that the teachers and the students that are teachers assistants are both direct and indirect stakeholders of the system . As it can also be seen, we did not find a great deal of different stakeholders, and we suspect that this could be because the amount of direct interaction with the system is really limited. 


\subsection{Benefits and harms}

This section shed a light on the benefits and harms for the system's stakeholders.
We recognized the different benefits and harms through a discussion following the definition of our stakeholders.

Benefits:

Direct stakeholders
- Students. They benefit from being given suggestions on behalf of their previous selected elective courses, which can then support them in finding their next elective course(s) that suit their preferences.
- Teachers. They benefit by being able to specify very precisely what the course is about and then getting students which have preferences for these kind of courses.
- Producers of non-keyword list. They benefit from the system being more effective in selecting the correct keywords.

Indirect stakeholders:
- Teachers. They benefit indirectly from the students being suggested courses according to their preferences, since the probability of the teachers getting motivated and interested students to their courses are heightened. They also benefit of students being recommended their courses, which will potentially help reach the minimum number of students of the course
- Teachers assistants. They benefit from the same as the teachers does, in that when the minimum student limit is reached, they have a job for the following semester.
- ITU. They benefit from the students being recommended courses from within ITU, which will potentially keep the student rate high on the ITU courses. Furthermore, they are potentially enhancing the students' overall welfare since the students possibly will attend an elective course they will enjoy.


Harms:

Direct stakeholders
- Students. The students are harmed in the way that they are recommended elective courses based on their previous selections of these, which might not be the most beneficial for them if there exist a course with a subject they had not yet interacted with that would be more interesting for them.
Furthermore, if they do not have any previous elective courses, the system will have no information to base its recommendations on.
Another harm could be if the student upon receiving the recommended elective courses, believes that the system is holidng a seat in the course, when it is the students own responsibility to add the course before is is fully booked.
- Teachers. They can potentially be harmed by the fact that they themselves have to update the course information. A poorly course description might lead to fewer recommendations of this course.
They can also be harmed if the producers of the non-keyword list adds a non-keyword that is essential to one or many of the teachers courses, since this will result in the course in some cases will not be taken into account.
- Producers of non-keyword list. 

Indirect stakeholders:
- Teachers.
- Teachers assistants.
- ITU. 




\subsection{Values}

In this section we will discuss how the benefits and harms mentioned in the previous section can affect the values we have decided to focus on for the ITU recommender system. 




Ownership and Property (Reason for choosing this value: ) 	(Benefits:     ; harms: if the student feel ownership over a course that has been recommended to him, )

Freedom from bias  (reason for choosing this value: in our experience we feel that when we start developing a project we always have an initial idea of how the users will interact with the system. If this idea is not tested in relation to the users, the system will ususally end up being biased.)		(Benefits:     ; harms: if the system only made a certain amount of recommendations in relation to the number of seats available then some students might be chosen over others (technical bias) also relevant for fairness, if the system does not make sure that all students get the recommendation mail at approximately the same time then some classes could in theory have no more available seats when a student read his mail (technical bias) also relevant for fairness, )

Diversity (Reason for choosing this value: Even though we are trying to give as precise recommendations as possible in relation the previous elective courses chosen by the student, we still want to bring an element of surprise into play, and maybe recommend a course that the student could be interested in even though it does not match with the students previous preferences)				(Benefits: If we are sending out ten recommendations to each student then two of them will be randomly chosen in relation to the "relevant for" tags in the course information (ensures that the randomly chosen ones are still relevant for the line of study that the student are at),    ; harms: the student is only recommended courses based on previous chosen ones)

Relevance (Reason for choosing this value: We are trying to ensure that in relation to elective courses at ITU, the students will be recommended new ones according to their previous preferences) 			(Benefits: The preference based recommendations will most likely ensure a big relevance of the recommended courses for the students,   ; harms: If we are sending out ten recommendations to each student then two of them will be randomly chosen in relation to the relevant tags in the course information (ensures that the randomly chosen ones are still relevant for the line of study that the student are at), )

Fairness (Reason for chosing this value: )				(Benefits:     ; harms: see two points under harms for freedom from bias)




\todo{embed freedom from bias - use already written text}
The freedom from bias will be the overall value we have focused on, trying to avoid any pre-existing, technical, or emergent biases. Maybe the text should be moved further up.


\todo{section on value conflicts}

describe value conflict between diversity and relevance --> leads to a design choice



