%!TEX root = ../../report.tex
\section{Conceptual}

\todo{move this part to chapter 2? - or keep it as it is?}

In accordance with the VSD-framework, this section we will describe how we practically identified the stakeholders for the system along with the benefits and harms the system can cause for these stakeholders. We will then continue to define which impact these benefits and harms will have on our chosen values. \newline

Stakeholders consists of individuals or groups that are directly or indirectly interacting with or affected by the system. Direct stakeholders are those individuals or groups that are interacting with the system in a direct manner or are subsequently affected by or interacting with the output of the system. Indirect stakeholders are the people who are affected by the system though they never directly interact with it. It is possible for an individual to be both directly and indirectly affected by the system, and hence being both a direct and indirect stakeholder.\newline
Benefits and harms defines the good and bad sides that the system have for the stakeholders of the system. In \citep[p.12]{FriedmanVSDandIS} it is described that it might be possible in some way to make every human an indirect stakeholder of the system, but here stresses the importance of prioritizing the individuals or groups that are most affected on an indirect level. \newline
Identifying the values that are affected by the benefits and harms are done to help determine whether a specific value has been successfully implemented into the system. 

\subsection{Stakeholders}

In order to find the stakeholders for the ITU recommender system, we sat down and tried to identify all people that were interacting with or affected by the system in any way. Below are the results we came up with. \newline

Direct stakeholders
\begin{itemize}
	\item Students.
They are interacting with the output of the system, and can be further divided into subgroups: Full time students (bachelors, masters) and part time students (bachelors, masters, single course students).
	\item Teachers.
Our initial thought was that the teachers are responsible for the course descriptions which defines the keywords for the individual courses, and hence have an impact on the system. We have subsequently discussed whether a teachers actually were a direct stakeholder, since our system only collects the data information from ITUs servers, and the teachers therefore do not have any direct interaction with the system. The reason we finally decided that they are part of the direct stakeholders is, that by changing the course description they will have the opportunity to alter the output of the system, hence we believe they should be seen as a direct stakeholder)
	\item Persons responsible for producing the non-keyword list.
They have are impacting the system in relation to the non-keywords they type in, since these can potentially alter the outcome of a keyword analysis, hence changing the relevance of a specific course.
\end{itemize}

Indirect stakeholders
\begin{itemize}
	\item Teachers.
They are indirectly impacted in regards to whether students choose to sign up for their course(s) or not, since the course can potentially be cancelled if there are not a minimum number of sign-ups.
	\item Teachers assistants.
Impacted indirectly in same way as teachers.
	\item ITU.
All the students can freely choose where they want to take their elective courses. ITU is indirectly impacted in the way that  through the system the students are encouraged to choose elective courses from within ITU. 
\end{itemize}

The above show us that the teachers and the students that are teachers assistants are both direct and indirect stakeholders of the system . As it can also be seen, we did not find a great deal of different stakeholders, and we suspect that this could be because the amount of direct interaction with the system is really limited. 


\subsection{Benefits and harms}

This section will shed a light on the benefits and harms for the system's stakeholders.
We recognized the different benefits and harms through a discussion following the definition of our stakeholders. \newline

Benefits for direct stakeholders
\begin{itemize}
	\item Students.
They benefit from being given suggestions on behalf of their previous selected elective courses, which can then support them in finding their next elective course(s) that suit their preferences.
	\item Teachers.
They benefit by being able to specify very precisely what the course is about and then getting students which have preferences for these kind of courses.
\item Producers of non-keyword list.
They benefit from the system being more effective in selecting the correct keywords. \todo{really?}
\end{itemize}

Benefits for indirect stakeholders
\begin{itemize}
	\item Teachers.
They benefit indirectly from the students being suggested courses according to their preferences, since the probability of the teachers getting motivated and interested students to their courses are heightened. They also benefit of students being recommended their courses, which will potentially help reach the minimum number of students of the course
	\item Teachers assistants.
They benefit from the same as the teachers does, in that when the minimum student limit is reached, they have a job for the following semester.
	\item ITU
They benefit from the students being recommended courses from within ITU, which will potentially keep the student rate high on the ITU courses. Furthermore, they are potentially enhancing the students' overall welfare since the students possibly will attend an elective course they will enjoy. \newline
\end{itemize}

Harms for direct stakeholders
\begin{itemize}
	\item Students.
The students are harmed in the way that they are recommended elective courses based on their previous selections of these, which might not be the most beneficial for them if there exist a course with a subject they had not yet interacted with that would be more interesting for them.
Furthermore, if they do not have any previous elective courses, the system will have no information to base its recommendations on.
Another harm could be if the student upon receiving the recommended elective courses, believes that the system is holding a seat in the course, when it is the students own responsibility to add the course before is is fully booked.
	\item Teachers.
They can potentially be harmed by the fact that they themselves have to update the course information. A poorly course description might lead to fewer recommendations of this course.
They can also be harmed if the producers of the non-keyword list adds a non-keyword that is essential to one or many of the teachers courses, since this will result in the course in some cases will not be taken into account.
	\item Producers of non-keyword list.
\end{itemize}

Harms for indirect stakeholders
\begin{itemize}
	\item Teachers.
	\item Teachers assistants.
	\item ITU.
\end{itemize}

As can be seen above, there are a variety of both benefits and harms that can affect the stakeholders of the system. In the next section we will have a look at our chosen values. \todo{in relation the the benefits and harms?}

\subsection{Values}

In this section we will first describe the values that we have chosen as the main values we will strive to implement into our ITU recommender system. Then we will discuss how the benefits and harms mentioned in the previous section can affect or are related to the values we have decided to focus on for the ITU system. 
How these values are implemented into our system will be explained later in this chapter. \newline
The three primary values we have chosen to focus on, are freedom from bias, relevance, and diversity. Beneath we have described what the meaning of each value is in relation to the ITU system, along with an explanation of the reason behind choosing each individual value. We will furthermore look at the values from a political and ethical perspective in order to seek out any implications the values can have in these areas.

\subsubsection{Freedom from bias}
\label{subsec:bias}
The meaning of freedom from bias is for us simply that the system we are developing does not contain any biases. In order to assist us with that in the development process, we have used some of the research from \todo{insert reference} on her framework about biases in order to get a better understanding of how to accomplish this. The reason we have chosen this value is, that in our experience of developing software projects, the developers all have an initial idea of how a user will use and interact with the system. These initial ideas can be correct, but it is also likely that the developers are somewhat biased in their perception. Our thoughts are that if these assumptions are not tested in direct relation to the users, the system has a risk of being biased. 
Freedom from bias is the overall value we have focused on for our project, because we believe that it is an ideal value to strive for in any system regardless of context. We have therefore chosen to go a bit deeper into the aspects of this value than the others. In order for us to better understand how to prevent the implementation of biases into the system, we will describe the three different types to be aware of in relation to \todo{insert reference} framework.

\begin{itemize}
	\item Pre-existing
A pre-existing bias describes a bias that in relation to the system exist by itself and in general before the system is created. A pre-existing bias can evolve from several different places, some of them being society as a whole, subcultures within the society, or via private or public institutions and organizations. They can furthermore be a reflection of personal biases and opinions from people closely related to the design process of the system, here putting a high emphasis on the client and system designer.
As mentioned, there are several areas from which a pre-existing bias can arise, and an important thing to remember is, that these can show themselves via a conscious effort from the people involved in the decision making process behind the system, but equally as well be integrated on an unconscious level, even when the system designers are trying their best to consciously avoid doing so. \todo{(refer to the study with boys/girls twice? Also in the HOW section}
	\item Technical
A technical bias describes a bias that is formed when the system designers are trying to resolve technical constraints or implement specific considerations. These biases can occur in relation to both hardware and software. Hardware-wise the biases are most likely getting fewer and fewer, mainly due to responsive webpages that throws away the need for specific monitor sizes and alike and at the same time giving the option of accessing property via alternative access devices such as tablets and smartphones. 
Software wise the biases can occur in a variety of ways. This includes algorithms that does not uphold a level of fairness to all users under all circumstances. This can be an algorithm that is used in the wrong context, for example if the system is designed to perform a specific task, and then at some point is used to solve a different kind of task. On the surface it might look like the system is solving the given assignment without any issues or implications, but underneath there could very well be some biases that would question the reliability of the system. \todo{example?}
Another way biases can arise in the context of software is when the designers are trying to implement a certain decision making part into the system that does not take human interpretation into context, but instead relies on a set of predefined formulas. This can possibly create some situations in which not everyone the system affects are treated in a fairly manner. \todo{bank credit example?}
	\item Emergent
An emergent bias can only arise in the context of use of a system. The emergent bias most often shows itself a great deal of time after a design has been made, and can be due to a change in the values regarding the system in question or another form of change in the users who utilizes the system. One place that seems to be prone to emergent biases is in the systems where user interfaces play a vital role. User interfaces are most often designed for a specific set of users. This could potentially create difficulties for a new set of users in the case that the system where eventually used in a different context, since they might not grasp the system functionality in the same way as the originally intended users can, thus creating an emergent bias in the system. 
\end{itemize}

The above points fairly describes the types of biases to be aware of when developing a \todo{only software} system. Below we have mentioned the benefit and harms that this value could bring with it. \newline

Benefits \newline
By consciously using the idea of eliminating biases as a part of the design process, we decided to remove some of our initial thoughts for the system functionality, and also thought about some important things that needed to be upheld, in order to better live up to this value. 
\begin{itemize}
	\item Our thought about this was that we should only send out a specific number of recommendations for a specific course, in relation to how many number of seats were available in the course. We chose to remove this feature because if the system only made a certain amount of recommendations in relation to the number of seats available, then some students could, for no apparent reason, be chosen over others. 
	\item If the system does not make sure that all students get the recommendation mail at approximately the same time then some classes could in theory have no more available seats when a student who received his recommendations late read his mail.
\end{itemize}

Harms
\begin{itemize}
	\item We have not found any harms by having this value implemented.
\end{itemize}

\todo{seems very related to fairness}



\subsubsection{Relevance}
The meaning of relevance for us in relation to the ITU system, is that the students will be given recommendations that are somewhat relevant to them, both on a personal as well as professional level. The personal level here indicates that the recommendations will be aligned with the students personal preferences in regard to previous chosen values. The professional level indicates that the recommendations will have relevance for the students line of study.

We have listed some benefits and harms of implementing this value into the ITU system. \newline \newline

Benefits
\begin{itemize}
	\item The preference based recommendations will most likely ensure a big relevance of the recommended courses for the students.
\end{itemize}

Harms
\begin{itemize}
	\item Since the students are only recommended courses based on previous chosen ones, this will remove the diversity of the recommender system.
\end{itemize}


\subsubsection{Diversity}
Diversity as a value in the ITU system means, that even though we are trying to give as precise recommendations as possible in relation to the previous elective courses chosen by the student, we still want to bring an element of surprise into play, and possibly recommend a course that the student could be interested in, even though it lies outside the students previous preferences. The level of diversity we will embed will stay inside the frames of the students line of study, to still maintain a high level of relevance. \newline

The benefits and harms we have found for implementing diversity into the ITU system. \newline \newline

Benefits
\begin{itemize}
	\item If the students are being recommended randomly chosen courses (still within the scope of the "relevant for" tags in the course information), this could enable the students to broaden their perspective with an potentially interesting course they would not initially have thought of.
\end{itemize}

Harms
\begin{itemize}
	\item If the students recommendations are based on diversity, there is a good chance that a lot of students will not get recommended courses that they deem relevant for either themselves or their line of study.
\end{itemize}


These above findings concludes our conceptual part of the VSD framework approach. The implementation of the values will be further elaborated in the technical part of this chapter.

\subsubsection{Political and ethical implications}

If a value-containing system becomes a standard in its field, then this or these values will become pervasive within its area, hence spreading and increasing its impact. This can in some cases have a profound impact seen from a technical, political, or ethical perspective. An example of this can be when a search based recommender system is set up to only give recommendations based on a users previous searches or interests. If the user in question tries to search for a specific political subject, the recommender system will here only give recommendations that are closely related to the users political interests. When this occurs, the user can possibly be mislead to believe that his or her own political convictions are omnipresent throughout society since the search based recommender system make it look like this point of view is the only one that exists. This can further make it harder for people to get their perspectives challenged since they in some way are �shown� that their opinions and meanings are the same ones shared by everyone else.

\todo{see points stated below to use in the HOW section}


%\todo{For the HOW section:} The freedom from bias value is hard to implement in a way that ensures that the system does not contain the slightest grain of bias. We have taken some precautions and steps that we believe will move us as close to implementation of the value as possible in our current situation.
%- We have removed the ability for the users, in this case the students, to interact with the system, hence ensuring that there will be no rise of an emergent bias in relation to the students using the system differently than intended.
%- We had a discussion about how a bias can be labeled as a bias, if individuals are aware of the fact that they are biased. Here we would like to refer to the research from \todo{insert reference} where she writes about a test where some individuals were to develop a game and were told to be aware of biases. The test showed, that even thought the control group believed they were aware of not being biased, they in fact still implemented some pre-existing biases into the system. 

%This led us to believe that you can give your best effort on the implementation of a value like freedom from bias, but it seems to us as if it is an ongoing process, or at least will be as long as the development of the system is ongoing. 

%emphasis on changing the �common� way systems are built, where biases are found and dealt with after the system has been developed, and instead working towards a mentality where we see the elimination of biases in software systems as an integrated part of the development process.


%---Bias is an ideal value to strive for--- though not as easy to accomplish due to emerging value conflicts
%We have tried to accomplish this by implementing the other values diversity and relevance


%See if I can find a political and ethical angle on the values






%\todo{section on value conflicts}
%Mention and describe the value conflict (contradiction) between diversity and relevance --> leads to a design choice


%Ownership and Property (Reason for choosing this value: ) 	(Benefits:     ; harms: if the student feel ownership over a course that has been recommended to him, )

%Fairness (Reason for choosing this value: )				(Benefits:     ; harms: see two points under harms for freedom from bias)




