%!TEX root = ../../report.tex

\section{Ending of chapter3}
\subsection{Value Considerations} % (fold)
\label{sub:value_considerations}
List of keywords vs list of non-keywords\\
In our current suggested system we are ignoring the case where students have been previously signed up for a course, but failed it. The system will still see this course as attended by the student, and will therefore remove it from the list of possible recommendations. This will in some sense make the system a bit biased, since we are not taking these courses into consideration. On the other hand, we also feel that a student who has attended a course at an earlier date, will be in a much better position to decide whether or not this course will be a future priority, than any current system will be able to.

Our initial thought regarding keywords for the courses was to get one or more well respected persons, whom in some way were related to the project, to sit down and choose these. Our idea was to make the leaders of the different lines of study at ITU create a list full of all the keywords they found relevant in relation to the fields of study. During our discussion we discovered that this approach could possibly lead to a biased keyword base. The reason for this is that the leaders, most likely unintentionally, all would believe that keywords related to their study were the most important. This could be solved by allowing all keywords to be added to the list, though this would raise another possible predicament, since one line of study might wish to include several keywords while another only have a few, which could tip the balance of the recommendations to one side. 
In order to fully get rid of these issues with pre-existing biases in the keywords, we decided that a person with no direct relation to the system should create a list of words that were NOT to be included in the search. This would be everyday words that would have no effect on the recommendation process. By using this approach we would ensure that the keywords are internal, chosen by the system itself in relation to the course content, while still keeping the runtime down and the biases away.
This does call for a high demand on precise notations of the course description for every course in order to be effective and ensure that our values are contained within the system. We do believe that this will be easier to maintain than keeping the keywords updated, since the leaders might change job positions and new ones have a different idea of which keywords to include, while the course descriptions will follow the actual content of the courses, which the recommendations are based on. 

\todo{I forhold til bias som en value, giv en indikation pŒ, at ved ikke at implementere brugerinteraktion i systemet, har vi taget et skridt mod at der ikke opstŒr emergent bias i vores system}

% subsection value_considerations (end)
\subsubsection{Future improvements} % (fold)
\label{sub:future_improvements}

% subsection future_improvements (end)

\subsection{End of chapter3} % (fold)
\label{sub:end_of_chapter3}

% subsection end_of_chapter3 (end)
\subsection{Discussion}
We know that in order to make considerate elective course recommendations for the students, the system will need some information or feedback on the users. The only current feedback available is from the students previous selected courses, which can help the system to give an indication of interests, though it is not a lot of information to build a thorough recommendation on when the data on the students elective courses are sparse. The lack of feedback from users is in general a problem for many recommender systems, being both ratings and content-feedback. In the cases where the student does not have any previous elective courses for the system to use in its calculations, we will in our approach cross reference the students line of study with the courses relevant for this specific line, and from here give some random recommendations. A future addition to the system could be to create a solution which take the students line of study into consideration. This could be done by cross referencing line of study with chosen elective courses from previous students, and hence give an indication of which recommendations might be relevant and useful for the individual students line of study. We acknowledge that this approach will give the system a slightly biased edge in relation to new courses which have never been previously selected, and therefore will not be included in these initial recommendations, hence we encourage future developers to seek out a solution for this as well, in the process of developing the system.\\
 
Using the VSD framework and finding values have been a fun and challenging experience. We started out thinking that it would be a somewhat straightforward linear process, but as it turned out it became more of an iterative approach. We started out like the framework itself suggested, by seeking to identify the relevant values for our project and bring them into the design process. However, this proved to be harder to do than we initially expected. Instead we found ourselves narrowing the list of proposed values from the framework, removing the values we did not feel had an inherent connection to our project. We did this without making any final selections on which of the remaining values to embed. We then continued our approach to design the system, and while doing this we kept the idea of 'value considerations in mind and noted whenever we were faced with a decision which could possibly affect the values in the system. \\
We believe that it is quite possible to develop a fully functional system without considering using the VSD framework for embedding the values into the system. An alternative to this developing approach could be to simply think of which values to implement in the system, in some cases maybe given in advance by the systems stakeholders. We do feel, however, that the fact that we had a variety of initial values helped us along in the process of both finding the values we felt was important to our system and also to keep one focused on the aspect of embedding values into the system in general.

We do not know the runtime of the system as it is proposed at this point, which will need to be tested when the system has been built. There is a chance when we are working with comparing several keywords (dimensions), that the time to calculate all the data will reach a critical threshold in relation to the number of dimensions.  We have done some research in order to seek out a solution in case this would become an apparent future issue to the system. A possible solution that we found is Locality Sensitive Hashing (LSH) \todo{REFERENCE TO ARTICLE FROM MIKKEL}. This works by "chopping" up the entire keyword space into small bits and assigning each of these with a hash value. It will then reduce the   comparison time by checking these hashed values with each other and hence giving an indication of similarity. We feel that this could be a solid alternative to the cosine similarity comparison method to give the system better speed at run time, in case the threshold is reached and this is getting too slow. According to the LSH study, the threshold for when it is most beneficial to switch is when the system has about 20 dimensions to compare at once.


\todo{TALK ABOUT USING REGEX TO GO THROUGH TEXT, OR WAIT WITH THAT UNTIL ITU PROJECT IN CHAPTER 3!?!?} 
